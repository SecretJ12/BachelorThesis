

\section{First Order Functions} \label{chapter:first_order}

We restrict our language to first-order functions without curried function applications for the first schema.
In other words, no function is allowed as an argument, and every function application has to evaluate the function entirely.

All found papers use quite similar schemas.
Therefore, we only look at two sources as they cover all the cases we want to cover in this work.
The scheme presented first was used by Sands in ``Calucli for Time Analysis of Functional Programs'' \parencite{sands}.
The schema is built up in the following way.
He starts by splitting up the functions into primitive (p) and non-primitive functions (f).
All primitive functions should only take a constant amount of time and, therefore, be cost-free for our evaluation.
They include basic mathematical operations and comparisons.
The cost for calling a non-primitive function is represented by its timing function.
As a result, we need to transform it by replacing it with its timing function instead.
In both cases, we also need to add the cost of the arguments.
Constants (c) are translated as $0$ because we are only interested in the time complexity.
In If-Else constructs, it would be possible to add up all the used terms,
but this would lead to a big overapproximation.
To get a better result, we only add up the cost for the condition, which is always evaluated, with the cost for the relevant branch.
The translation function is named $\mathcal{T}$ and is described in figure \ref{fig:first_order_T}.

When converting a function definition, we start with $1$ for the function call itself.
Afterward, the expression converted by $\mathcal{T}$ is added.
The conversion function is named $\mathcal{C}$ and is described in figure \ref{fig:first_order_C}.
Sands also provides a proof for this schema.
Chapter \ref{chapter:formal} presents a formalization of the proof in Isabelle.

\begin{figure}
  \begin{align*}
    \mathcal{T}\llbracket f\ a_{1}\ \dots\ a_{n}\rrbracket &= (T\_f\ a_{1}\ \dots\ a_{n} + \mathcal{T}\llbracket a_{1}\rrbracket +\ \dots\ + \mathcal{T}\llbracket a_{n}\rrbracket)\\
    \mathcal{T}\llbracket p\ a_{1}\ \dots\ a_{n}\rrbracket &= (\mathcal{T}\llbracket a_{1}\rrbracket +\ \dots\ + \mathcal{T}\llbracket a_{n}\rrbracket)\\
    \mathcal{T}\llbracket c\rrbracket &= 0\\
    \mathcal{T}\llbracket \texttt{IF}\ c\ \texttt{THEN}\ e_{1}\ \texttt{ELSE}\ e_{2}\rrbracket &= (f\ a_{1}\ \dots\ a_{n} = 1 + \mathcal{T}\llbracket e\rrbracket)
  \end{align*}
  \caption{Translation function $\mathcal{T}$ for expressions of first-order functions}
  \label{fig:first_order_T}
\end{figure}

\begin{figure}
\begin{align*}
    \mathcal{C}\llbracket f\ a_{1}\ \dots\ a_{n} = e\rrbracket = (f\ a_{1}\ \dots\ a_{n} = 1 + \mathcal{T}\llbracket e \rrbracket)
\end{align*}
\caption{Translation function $\mathcal{C}$ for function definitions of first-order functions}
\label{fig:first_order_C}
\end{figure}

The schema introduced by Nipkow in ``Functional Data Structures'' provides the same translation for the cases presented so far \parencite{fds}.
Additionally, he defines a schema to translate case and let expressions.
We look at them here as we will need them for the implementation.
Similar to if-else constructs, we avoid too much overapproximation by only counting the executed case.
To this, we add the running time equivalent of the pattern-matching expression.
For the let expression, we add the running time version of the assigned expression to the let expression with the translated body.
If the translated body does not use the assigned variable, we can reduce the construct to the body.
The schema can be found in listing \ref{lst:fds_case_let}.
\begin{lstlisting}[language=translation,label=lst:fds_case_let,caption=Translation schema for case- and let-expression by Nipkow,mathescape=true]
  cTb$$case $e$ of $p_{1}$ => $e_{1}$ | $\dots$ | $p_{k}$ => $e_{k}\rrbracket$
    = cTb$ e\rrbracket$ + (case $e$ of $p_{1}$ => cTb$e_{1} \rrbracket$ | $\dots$ | $p_{k}$ => cTb$e_{k} \rrbracket$)
  cTb$$let $x$ = $e_{1}$ in $e_{2}\rrbracket$ = cTb$e_{1}\rrbracket$ + (let $x$ = $e_{1}$ in cTb$e_{2} \rrbracket)$
\end{lstlisting}
