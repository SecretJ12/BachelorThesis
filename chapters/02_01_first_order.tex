
\section{First Order Functions}

At first we gonna look at First Order Function.
Here, no function are allowed as arguments.
Also the use of curried functions is forbidden, this means every function has to be fully evaluated in one application.

The first scheme presented first was used by Sands in ``Calucli for Time Analysis of Functional Programs'' \parencite{sands}.
His key idea was splitting up the functions into primitive (p) and non-primitive functions (f).
Als primitive function should be free of cost, this should include basic mathematical operations and comparisons.
When calling non-primitive function the timing function should be evaluated instead.
In both cases also the arguments need to be transformed and added.
Constants (c) are also seen as free and therefore translated as 0.
In If-Else constructs it would be possible to just add up all the used terms,
but this would lead to a big overapproximation.
To get a better result, the relevant branch along with the condition is counted.
The function translating this the expressions is named $\mathcal{T}$.

When converting a function definition, we start with $1$ for the function call itself.
Afterwards the expression converted by $\mathcal{T}$ is added.
The conversion function is named $\mathcal{C}$.
The full schema for translating expressions is shown in listing \ref{lst:basic_sands}.
Sands also provides as proof for this schema. A formalization can be found in chapter \ref{chapter:formal}.

\begin{lstlisting}[language=translation,label=lst:basic_sands,caption=Translation schema for first order functions by Sands,mathescape=true]
  cTb$ f$ $a_{1}$ $\dots$ $a_{n}\rrbracket$ = ($T\_f$ $a_{1}$ $\dots$ $a_{n}$ + cTb$a_{1} \rrbracket$ + $\dots$ + cTb$a_{n}\rrbracket$)
  cTb$ p$ $a_{1}$ $\dots$ $a_{n}\rrbracket$ = (cTb$a_{1} \rrbracket$ + $\dots$ + cTb$a_{n}\rrbracket$)
  cTb$ c \rrbracket$ = $0$
  cTbIF $c$ THEN $e_{1}$ ELSE $e_{2}\rrbracket$ = (cTb$c \rrbracket$ + IF $c$ THEN cTb$e_{1} \rrbracket$ ELSE cTb$e_{2} \rrbracket$)

  cCb$f$ $a_{1}$ $\dots$ $a_{n}$ = $e\rrbracket$ = ($f$ $a_{1}$ $\dots$ $a_{n}$ = $1$ + cTb$e \rrbracket$)
\end{lstlisting}

The schema introduced by Nipkow in ``Functional Data Structures'' slightly different approach regarding primitive functions and constants \cite{fds}.
At first every primitive function, here defined as basic mathematical operations and constructors, as well as constants are translated as 1.
In a simplification step all $+ 1$ except one are dropped, as they don't change the asymptotic running time.
This results in the same schema as given by Sands.
Additionally it gives a schema to translate case and let expressions.
Case works similar to if-else, while the body of let expressions are being converted to lambdas. (\#TODO this might change)
The schema can be found in listing \ref{lst:fds_case_let}.
\begin{lstlisting}[language=translation,label=lst:fds_case_let,caption=Translation schema for case- and let-expression by Nipkow,mathescape=true]
  cTb$$case $e$ of $p_{1}$ => $e_{1}$ | $\dots$ | $p_{k}$ => $e_{k}\rrbracket$
    = cTb$ e\rrbracket$ + (case $e$ of $p_{1}$ => cTb$e_{1} \rrbracket$ | $\dots$ | $p_{k}$ => cTb$e_{k} \rrbracket$)
  cTb$$let $x$ = $e_{1}$ in $e_{2}\rrbracket$ = cTb$e_{1}\rrbracket$ + (let $x$ = $e_{1}$ in cTb$e_{2} \rrbracket)$
\end{lstlisting}
