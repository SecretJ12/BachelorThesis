
\section{Higher Order functions} \label{chapter:higher_order}

In this chapter, we want to extend the schema for higher-order functions.
First, we look at the function $\texttt{test}$ in \autoref{lst:ho_test}.
\begin{lstlisting}[language=isabelle,mathescape=true,label=lst:ho_test,caption=Example function]
  fun test :: (nat => bool) => nat => nat where
    test f a = (if f a then 1 else 0)
\end{lstlisting}
Applying the translation $\mathcal{T}$ for first-order functions onto the expression yields us
\begin{lstlisting}[language=isabelle,mathescape=true]
  T_f a + (if f a then 0 else 0).
\end{lstlisting}

This example shows that we need a way to address the original function as well as the timing function.
As we cannot generate the timing function at runtime, Sands proposes to change every function argument by a pair \parencite{sands}.
The first element is the original function and the second is the timing function.
Using this idea, we can convert the whole function as shown in \autoref{lst:ho_T_test}.
\begin{lstlisting}[language=isabelle,mathescape=true,label=lst:ho_T_test,caption=Timing function of example function]
  fun T_test :: ('a => bool) $\times$ ('a => nat) => 'a => nat where
    T_test f a = 1 + snd f a + (if fst f a then 1 else 0)
\end{lstlisting}
In general, we need to use $\texttt{snd}$ everywhere the timing function is needed and $\texttt{fst}$ in case of a normal evaluation.
For every function $\texttt{f}$ he generates two functions $\texttt{f'}$ and $\texttt{cf}$.
The function $\texttt{f'}$ evaluates to the result of the original function but with the described pair instead of functions as an argument.
The function $\texttt{cf}$ is the timing function, called cost function by Sands.
The translation schema is described in \autoref{fig:higher_transl}.
\begin{figure}
  \begin{align*}
  f'\ x_{1}\ \dots\ x_{n} &= \mathcal{V}\llbracket e \rrbracket \\
  cf\ x_{1}\ \dots\ x_{n} &= 1 + \mathcal{N}o\mathcal{V}\llbracket e \rrbracket
  \end{align*}
  \caption{Translation schema for higher-order function with expression $e$}
  \label{fig:higher_transl}
\end{figure}

He uses the two referenced functions $\mathcal{V}$ and $\mathcal{T}$ for conversion.
The function $\mathcal{V}$ replaces every function identifier by the described pair.
Therefore, it ensures that every function will be called with the expected pair.
The function $\texttt{fst}$ is used for every function application.
The result behaves as described for the function $f'$.
The schema is described in \autoref{fig:higher_V}.
For the timing function, the function $\mathcal{T}$ is applied afterward.
It behaves similarly to the schema described for the first-order function in \autoref{fig:first_order_T}.
The only difference is that the $\texttt{snd}$ function is used in every application to get the timing function.
The schema is written in \autoref{fig:higher_T}.
Function definitions will not be converted to a pair and back by those functions.
This optimization step could be dropped but keeps the timing function readable.
\begin{figure}
  \begin{align*}
    \mathcal{V}\llbracket f\ a_{1}\ \dots\ a_{n}\rrbracket &= f'\ \mathcal{V}\llbracket a_{1}\rrbracket\ \dots \ \mathcal{V}\llbracket a_{n}\rrbracket\\
    \mathcal{V}\llbracket p\ a_{1}\ \dots\ a_{n}\rrbracket &= p\ \mathcal{V}\llbracket a_{1}\rrbracket\ \dots \ \mathcal{V}\llbracket a_{n}\rrbracket\\
    \mathcal{V}\llbracket c \rrbracket &= c\\
    \mathcal{V}\llbracket \text{IF}\ c\ \text{THEN}\ e_{1}\ \text{ELSE}\ e_{2}\rrbracket &= \text{IF}\  \mathcal{V}\llbracket e_{1}\rrbracket\ \text{THEN}\ \mathcal{V}\llbracket e_{2}\rrbracket\ \text{ELSE}\ \mathcal{V}\llbracket e_{2}\rrbracket\\
    \mathcal{V}\llbracket f\rrbracket &= (f',cf)\\
    \mathcal{V}\llbracket p\rrbracket &= (p,cp)\\
    \mathcal{V}\llbracket e\ a_{1}\ \dots\ a_{n}\rrbracket &= \texttt{fst}\ \mathcal{V}\llbracket e\rrbracket\ \mathcal{V}\llbracket a_{1} \rrbracket\ \dots\ \mathcal{V}\llbracket a_{n}\rrbracket
  \end{align*}
  \caption{Translation schema V for higher-order functions}
  \label{fig:higher_V}
\end{figure}
\begin{figure}
  \begin{align*}
    \mathcal{T}\llbracket f'\ a_{1}\ \dots\ a_{n}\rrbracket &= \mathcal{T}\llbracket a_{1}\rrbracket + \dots + \mathcal{T}\llbracket a_{n}\rrbracket + cf\ a_{1}\ \dots\ a_{n}\\
    \mathcal{T}\llbracket p\ a_{1}\ \dots\ a_{n}\rrbracket &= \mathcal{T}\llbracket a_{1}\rrbracket + \dots + \mathcal{T}\llbracket a_{n}\rrbracket\\
    \mathcal{T}\llbracket c \rrbracket &= 0\\
    \mathcal{T}\llbracket \text{IF}\ c\ \text{THEN}\ e_{1}\ \text{ELSE}\ e_{2}\rrbracket &= \mathcal{T}\llbracket c\rrbracket + \text{IF}\  c\ \text{THEN}\ \mathcal{V}\llbracket e_{1}\rrbracket\ \text{ELSE}\ \mathcal{T}\llbracket e_{2}\rrbracket\\
    \mathcal{T}\llbracket\texttt{fun}\ e\ a_{1}\ \dots\ a_{n}\rrbracket &= \mathcal{T}\llbracket e\rrbracket + \mathcal{T}\llbracket e_{1}\rrbracket + \dots + \mathcal{T}\llbracket e_{n}\rrbracket + \texttt{cost}\ e\ e_{1}\ \dots\ e_{n}\\
    \mathcal{T}\llbracket (f',cf)\rrbracket &= 0\\
    \mathcal{T}\llbracket (p,cp)\rrbracket &= 0
  \end{align*}
  \caption{Translation T for higher-order functions}
  \label{fig:higher_T}
\end{figure}

Nipkow provides no schema for higher-order functions.
However, as he also needs some higher-order functions for the analysis, he provides a translation.
In \autoref{lst:T_map_nipkow}, the translation for the $\texttt{map}$ function is shown.
It differs from our schema as it only takes the timing function but not the original function.
This schema works as the original function is not used but cannot be extended to all functions.
\begin{lstlisting}[language=isabelle,mathescape=true,caption=Translation of the function map to their timing function by Nipkow,label=lst:T_map_nipkow]
  fun map :: ('a => 'b) => 'a list => 'b list where
    map f [] = []
  | map f (x # xs) = f x # map f xs

  fun T_map :: ('a => nat) => 'a list => nat where
    T_map f [] = 1
  | T_map f (x # xs) = 1 + (f x + T_map f xs)
\end{lstlisting}
