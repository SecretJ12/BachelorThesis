
\section{Curried Functions} \label{chapter:rel_curried}

Sands also specifies a schema for curried functions.
It operates assuming that the application is free of cost as long as the function is not fully applied.
The key point of this schema is the extension of the pair used for higher-order function arguments to a triple.
The third argument represents the number of arguments left to apply till full application.
This third argument is called the arity.
He explains the schema by demonstrating it on a default apply function first.
The apply function $\texttt{app}$ is shown in \autoref{lst:curried_app}.
\begin{lstlisting}[language=isabelle,mathescape=true,caption=Apply function,label=lst:curried_app]
  fun app :: ('a => 'b) => 'a => 'b where
    app f a = f a
\end{lstlisting}

He also starts by defining the function $\texttt{app'}$.
This function accepts the described triple instead of the function.
If the arity equals $1$, then the curried function expects only one more argument.
Then, the first element of the tuple representing the original function is evaluated.
If the arity is bigger than 1, the curried function will evaluate to another function.
Therefore, we construct a new tuple with the argument applied to both functions and the arity decreased by $1$.
The full function is described in \autoref{lst:curried_app'}.
\begin{lstlisting}[language=isabelle,mathescape=true,label=lst:curried_app',caption=Apply function on function argument triple]
  fun app' where
    app' f x = (if arity f = 1
        then (func f x)
        else (func f x, cost f x, arity f - 1))
\end{lstlisting}

Afterward, he defines the function $\texttt{capp}$.
It should yield the cost of the passed function with one argument applied.
As the application is free if the function is not fully applied, it yields $0$ if the arity is greater than $1$.
If the arity is $1$, then the cost function is evaluated.
This gives the function definition shown in \autoref{lst:curried_capp}.
\begin{lstlisting}[language=isabelle,mathescape=true,caption=Timing function of apply function,label=lst:curried_capp]
  fun capp where
    capp f x = (if arity f = 1 then cost f x else 0)
\end{lstlisting}

Similar to the translation of the higher-order function, he defines two conversion functions $\mathcal{V}$ and $\mathcal{T}$ with similar behavior.
$\mathcal{V}$ converts the function to evaluate as the original function but accepts the described triples instead of functions.
Every referenced function will be replaced by the described triple.
Function applications are replaced by passing function and argument to the $\texttt{app'}$ function.
All the other definitions stay the same as described in \autoref{fig:higher_V}.
The function $\mathcal{T}$ also stays the same as in \autoref{fig:higher_T} except for the application case.
For every application currently handled by the $\texttt{app'}$ function, the function and the argument will be converted and added up.
Additionally, we need the cost of the application itself.
This cost is determined by passing the function and the argument to the function $\texttt{capp}$.
\autoref{fig:curried_T_apply} shows this case.
For better readability, the infix symbol $@$ represents the function $\texttt{app'}$ while $c@$ stands for the function $\texttt{capp}$.
For example, we can see the transformation of the map function in listing \ref{lst:curried_map_example}.
\begin{figure}
\begin{align*}
  \mathcal{T}\llbracket f\ @\ a\rrbracket = \mathcal{T}\llbracket f\rrbracket + \mathcal{T}\llbracket a\rrbracket + (f\ c@\ a)
\end{align*}
\caption{Converting application in case of curried higher-order functions}
\label{fig:curried_T_apply}
\end{figure}

\begin{lstlisting}[language=isabelle,mathescape=true,label=lst:curried_map_example,caption=Example translation for map function]
fun map where
  map f [] = []
| map f (x#xs) = f x # map f xs

fun map' where
  map' f [] = []
| map' f (x#xs) = (Cons,T_Cons,2) @ (f @ x) @ ((map',T_map,2) @ f @ xs)

fun T_map where
  T_map f [] = 1
| T_map f (x#xs) = 1 + f c@ x +
    (map',T_map,2) c@ f + (map',T_map,2) @ f c@ xs +
    (Cons,T_Cons,2) @ (f @ x) c@ ((map',T_map,2) @ f @ xs)
\end{lstlisting}

As we can see, this already blows up for a small function as $\texttt{map}$.
To overcome this issue, Sands introduces some basic optimization steps.
If the $\texttt{capp}$ function is applied onto a triple, an application to a defined function happens.
We differ between two cases here.
If the arity is greater than $1$, it can be dropped as it would evaluate to $1$ either.
In case the arity is $1$, we can replace this by the application to the cost function.
For the normal application $\texttt{app'}$, he applies the same but evaluates the normal function instead of the cost function.
As a result, we gain a timing function similar to our previous schema.
The apply function will only be used for passed functions now.
\begin{lstlisting}[language=isabelle,mathescape=true]
function T_map where
  T_map f [] = 1
| T_map f (x#xs) = 1 + f c@ x + T_map f xs
\end{lstlisting}
