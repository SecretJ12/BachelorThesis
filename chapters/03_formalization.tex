% !TeX root = ../main.tex
% Add the above to each chapter to make compiuling the PDF easier in some editors.

\chapter{Formalization}\label{chapter:formal}

Sands provides a correctness proof for all translation schemas presented in chapter \ref{chapter:relwork}.
This chapter looks at the correctness proof for a first-order language.
The proof was formalized in Isabelle and can be found in the associated GitHub repository [\#TODO link].
The term cost function is used interchangeably for timing function here.

We start by defining the first-order imperative language.
The datatype $\texttt{val}$ represents the type our language operates over.
It needs to fulfill two requirements.
The language should work with arbitrary types we do not want to define yet.
In order to count the number of function calls, we need a type that can count them.
Therefore, we define a type $\texttt{value}$, which is not further specified.
This represents our arbitrary type.
Based on this, we define the constructors for our datatype.
The $\texttt{V}$ constructor maps the arbitrary type and the $\texttt{N}$ constructor wraps the natural numbers we will use for counting.
Therefore it fulfills our requirements.
\begin{lstlisting}[language=isabelle]
typedecl "value"
datatype val = N nat | V "value"
\end{lstlisting}

For if-else constructs, we need a boolean interpretation of our datatype.
We define the natural number type with 0 as false and everything else as true.
\begin{lstlisting}[mathescape=true,language=isabelle]
definition false :: "val => bool" where "false v $\equiv$ v = N 0"
definition true :: "val => bool" where "true v $\equiv$ $\neg$ false v"
\end{lstlisting}

As explained in the last chapter, we differ between primitive and non-primitive functions.
The call of a primitive function is free, while a non-primitive function call costs 1.
In the following, we will refer to non-primitive functions only as functions.
For each of them, we now define a datatype representing an identifier.
For functions, we also need to differentiate between an identifier for a function and its respective cost function.
This distinction avoids collisions in the namespace for the final correctness proof.
\begin{lstlisting}[language=isabelle]
datatype funId =
  Fun string
| cFun string
datatype pfunId = pFun string
\end{lstlisting}

We can now define expressions.
An application takes a list of arguments, each represented by another expression.
For better readability, we introduce \$ for function applications and \$\$ for primitive function applications.
The whole datatype can be found in listing \ref{lst:exp}.
\begin{lstlisting}[label=lst:exp,caption=Expression syntax,float,language=isabelle,escapeinside={@}{@}]
datatype exp =
  App funId "exp list"        (infix "@\$@" 100)
| pApp pfunId "exp list"      (infix "@\$\$@" 100)
| If exp exp exp              ("(IF _/ THEN _/ ELSE _)")
| Ident nat
| Const val
\end{lstlisting}

The datatype storing all registered functions is called $\texttt{defs}$.
It is a simple function mapping a function id to an expression wrapped by an option.
The option is again to deal with possible namespace collisions later.
\begin{lstlisting}[mathescape=true,language=isabelle]
type_synonym defs = "funId => exp option"
\end{lstlisting}

For primitive functions, there is no mapper but a direct primitive application function.
It takes the name of a primitive function and a list of arguments.
The application function should be universal in order to support any kinds of primitive functions on different machines.
As the translation schema demands addition, we assume the function $\texttt{sum}$ to be defined.
All the arguments need to be of the defined number type.
We can assume this function to be constant, although it takes an arbitrary amount of arguments.
This choice is justified as the number of arguments is fixed in every expression.
\begin{lstlisting}[mathescape=true,language=isabelle]
axiomatization pApp :: "string => val list => val" where
  sum: "pApp ''sum'' es = (N o sum_list o map val_to_nat) es"
\end{lstlisting}

Inside a function, the passed arguments will be available as local variables.
Those local variables are contained in the environment, represented by a list of values.
We can refer to a value inside an expression through $\texttt{Ident}$.
The selected number represents the index of the environment list.
\begin{lstlisting}[language=isabelle]
type_synonym env = "val list"
\end{lstlisting}

The evaluation semantic of the form
\begin{equation*}
  \rho, \phi \hvdash \text{e} \hrightarrow \text{v}
\end{equation*}
can be read as \textit{The expression $\text{e}$ evaluates under the local variables $\rho$ and the definitions $\phi$ to the value $\text{v}$}. The corresponding semantic is given in figure \ref{lst:eval}.
\begin{lstlisting}[float,label=lst:eval,caption=Default evaluation semantic,mathescape=true,escapeinside={@}{@},language=isabelle]
inductive eval :: "env => defs => exp => val => bool"
  ("(_/, _/ |-_/ -> _)") where
Id:   "$\rho$, $\phi$ |- Ident i -> ($\rho$ ! i)" |
C:    "$\rho$, $\phi$ |- Const v -> v" |
F:    "length es = length vs
  ==> ($\forall$i < length vs. $\rho$, $\phi$ |- (es ! i) -> (vs ! i))
  ==> $\phi$ f = Some fe ==> vs, $\phi$ |- fe -> v
  ==> $\rho$, $\phi$ |- (f@\$@ es) -> v" |
P:    "length es = length vs
  ==> ($\forall$i < length es. $\rho$, $\phi$ |- (es ! i) -> (vs ! i))
  ==> pApp p vs = v ==> $\rho$, $\phi$ |- (pFun p@\$\$@ es) -> v" |
If1:  "$\rho$, $\phi$ |- b -> v ==> true v
  ==> $\rho$, $\phi$ |- t -> et ==> $\rho$, $\phi$ |- (IF b THEN t ELSE f) -> et" |
If2:  "$\rho$, $\phi$ |- b -> v ==> false v
  ==> $\rho$, $\phi$ |- f -> ef ==> $\rho$, $\phi$ |- (IF b THEN t ELSE f) -> ef"
\end{lstlisting}

For the eval semantic, we can show determinism stated by
\begin{equation*}
 \llbracket \rho, \phi \hvdash e \hrightarrow v; \text{ } \rho, \phi \hvdash e \hrightarrow v' \rrbracket \Longrightarrow v = v'.
\end{equation*}

The proof is an induction over eval, which can be solved mostly automatically by metis and blast.
In the next step, we define a step-counting semantic of the form
\begin{equation*}
  \rho, \phi \hvdash e \hrightarrows (v,t)
\end{equation*}

which is read as \textit{The expression $\text{e}$ evaluates under the given local variables $\rho$ and the definitions $\phi$ to the value $\text{v}$ involving t non-primitive function applications}. The corresponding semantic is given in figure \ref{lst:stepeval}.
\begin{lstlisting}[float,label=lst:stepeval,caption=Step-Counting Semantic,mathescape=true,escapeinside={@}{@},language=isabelle]
inductive eval_count :: "env => defs => exp => val * nat => bool"
  ("(_/, _/ |- _/ ->s _)") where
cId:   "$\rho$, $\phi$ |- Ident i ->s ($\rho$!i,0)" |
cC:    "$\rho$, $\phi$ |- Const v ->s (v,0)" |
cF:    "length es = length vs ==> length es = length ts
  ==> ($\forall$ i < length vs. $\rho$, $\phi$ |- (es!i) ->s (vs!i,ts!i))
  ==> $\phi$ f = Some fe ==> vs, $\phi$ |- fe ->s (v,t)
  ==> $\rho$, $\phi$ |- (f@\$@ es) ->s (v,1+t+sum_list ts)" |
cP:    "length es = length vs ==> length es = length ts
  ==> ($\forall$i < length es. $\rho$, $\phi$ |- (es ! i) ->s (vs!i,ts!i))
  ==> pApp p vs = v ==> $\rho$, $\phi$ |- (pFun p@\$\$@ es) ->s (v,sum_list ts)" |
cIf1:  "$\rho$, $\phi$ |- b ->s (eb,tb) ==> true eb ==> $\rho$, $\phi$ |- t ->s (et,tt)
  ==> $\rho$, $\phi$ |- (IF b THEN t ELSE f) ->s (et,tb+tt)" |
cIf2:  "$\rho$, $\phi$ |- b ->s (eb,tb) ==> false eb ==> $\rho$, $\phi$ |- f ->s (ef,tf)
  ==> $\rho$, $\phi$ |- (IF b THEN t ELSE f) ->s (ef,tb+tf)"
\end{lstlisting}

We now want to show the equivalence of the semantics according to the resulting value.
Showing the implication from the step-counting semantic to the normal semantic can be done by induction over the induction schema of the step-counting semantic.
The auto tactic can handle all cases.
\begin{lstlisting}[language=isabelle,mathescape=true]
  lemma eval_count_eval: "$\rho$, $\phi$ |- b ->s (v,t) ==> $\rho$, $\phi$ |- b -> v"
  by (induction $\rho$ $\phi$ b "(v,t)" arbitrary: v t rule: eval_count.induct) auto
\end{lstlisting}

For the other direction, we need a lengthy auxiliary lemmo to deal with the arguments of function applications.
It states the following:
\begin{lstlisting}[language=isabelle,escapeinside={@}{@},mathescape=true]
lemma eval_eval_count':
  assumes tex: "$\forall$i < length vs. $\exists$t. $\rho$, $\phi$ |- es ! i ->s (vs ! i, t)"
    and $\text{   }$@ @len: "length es = length vs"
   shows  @ @"$\exists$ts. length vs = length ts
             $\land$ ($\forall$i. (i < length vs $\longrightarrow$ $\rho$, $\phi$ |- (es!i) ->s (vs!i,ts!i)))"
\end{lstlisting}

The lemma is shown as an induction over the length of $\texttt{vs}$ with arbitrary es using the assumptions.
While the simp tactic can handle the Nil case, the Cons case needs more effort.
We get the assumption tex as premise for $\texttt{v\#vs}$.
We can easily show that this implies the same for $\texttt{vs}$.
With this result, we can now use the IH and obtain the $\texttt{ts}$ wanted for the $\texttt{vs}$.
We can obtain $\texttt{t}$ for $\texttt{v}$ from the assumption of tex.
Combining this gives the second part of the result.
The length equality can be seen easily from here, which completes the proof.

With the help of this lemma it is now possible to show the direction from eval semantic to step-counting semantic by induction of the eval semantic.
These results give us the equivalence of the two semantics according to the evaluation result.

Sands also claims in a proposition that the counting result of the step-counting semantic equals the count the rule F was used in the eval semantic.
This setup is quite laborious in Isabelle while only providing a proof for an easy-to-see step.
Therefore, we neglect this proof.

We can now define the main function, which converts an expression to its cost function.
The used schema corresponds to the one presented in \ref{chapter:first_order}.
\begin{lstlisting}[language=isabelle,mathescape=true,escapeinside={@}{@},caption=Translation function for expressions]
fun $\mathcal{T}$ :: "exp => exp" where
  "$\mathcal{T}$ (Const v) = Const (N 0)"
| "$\mathcal{T}$ (Ident i) = Const (N 0)"
| "$\mathcal{T}$ (IF b THEN t ELSE f)
   = pFun ''sum''@\$\$@ [$\mathcal{T}$ b, IF b THEN $\mathcal{T}$ t ELSE $\mathcal{T}$ f]"
| "$\mathcal{T}$ (pFun @\_\$\$@ args) = pFun ''sum''@\$\$@ map $\mathcal{T}$ args"
| "$\mathcal{T}$ (Fun f@\$@ args) = pFun ''sum''@\$\$@ (cFun f@\$@ args @\#@ map $\mathcal{T}$ args)"
\end{lstlisting}

Now, we add a definition to convert a function expression.
It uses the $\mathcal{T}$ function and adds $1+$ to the front representing the function call.
We also define the function $\texttt{conv}$ based on this function.
It alters a definition and registers the cost function for each defined function.
Already defined cost functions will be dropped.
\begin{lstlisting}[language=isabelle,mathescape=true,escapeinside={@}{@}]
definition cost :: "exp => exp" where
  "cost e = pFun ''sum''$$ [Const (N 1), $\mathcal{T}$ e]"

fun conv :: "defs => defs" where
  "conv $\phi$ (Fun f) = $\phi$ (Fun f)"
| "conv $\phi$ (cFun f) = (case $\phi$ (Fun f) of None => None
                                           | Some e => Some (cost e))"
\end{lstlisting}

For the final correctness theorem, we need a property claiming that no timing function is defined in a given definition.
This property avoids errors of already referenced timing functions being dropped by the conv function.
\begin{lstlisting}[language=isabelle,mathescape=true,escapeinside={@}{@}]
definition no_time where
  "no_time $\phi$ = ($\forall$f. $\phi$ (cFun f) = None)"
\end{lstlisting}

In order to better deal with this property, we define another auxiliary lemma.
It states that if there is no timing function, then the evaluation result does not change when the expression is converted using $\texttt{conv}$.
The lemma is proved over induction with a helping lemma claiming conv does not change defined functions.
\begin{lstlisting}[language=isabelle,mathescape=true,escapeinside={@}{@}]
lemma no_time_trans:
  "no_time $\phi$ ==> $\phi$ f = Some e ==> (conv $\phi$) f = Some e"
  by (cases f) (auto simp: no_time_def)
lemma eval_no_time_trans:
  "$\rho$, $\phi$ |- e -> v ==> no_time $\phi$ ==> $\rho$, (conv $\phi$) |- e -> v"
  by (induction rule: eval.induct) (auto simp: no_time_trans)
\end{lstlisting}

We can now show the final correctness theorem.
Given an expression and a definition without a timing function.
If an evaluation of the step-counting semantics results in $\texttt{t}$, then the evaluation of the converted expression and function definitions results in the same value $\texttt{t}$.
\begin{lstlisting}[language=isabelle,mathescape=true,escapeinside={@}{@},keepspaces=true]
theorem conv_cor:
  assumes "$\rho$, $\phi$ |-e ->s(s,t)"
    and   "no_time $\phi$"
   shows  "$\rho$, (conv $\phi$) |-($\mathcal{T}$ e) ->N t"
\end{lstlisting}

The theorem is proved by induction over the step-counting semantic.
While the simplest cases, Id and C, can be solved automatically, we need an Isar proof for the other cases.
In the following, we look at the most important case F, which represents the application of a function.
We have the following assumptions:
\begin{align}
  &\text{length es = length vs = length ts} \label{cor:len}\\
  &\forall\text{i < length vs. } (\rho, \phi \hvdash \text{es ! i} \hrightarrows \text{es ! i} \hrightarrow \text{N (ts ! i)}\ \land \label{cor:args}\\
  &\ \ (\text{no\_time} \phi \longrightarrow \rho, \text{conv} \phi \hvdash \mathcal{T} \text{(es ! i)} \hrightarrow \text{N (ts ! i)}))\nonumber\\
  &\phi\ \text{f} = \text{Some fe} \label{cor:phi}\\
  &\text{vs}, \phi \hvdash \text{fe} \hrightarrow \text{(v,t)} \label{cor:feeval}\\
  &\text{no\_time}\ \phi \label{cor:notime}\\
  &\text{vs, conv}\ \phi \hvdash \mathcal{T}\ \text{fe} \hrightarrow \text{N t} \label{cor:Tfeeval}
\end{align}

\ref{cor:Tfeeval} is already collapsed as the premise is given by \ref{cor:notime}.
We need to show
\begin{equation*}
  \rho, \text{conv}\ \phi \vdash\mathcal{T} (\text{f\$ es})\rightarrow \text{N} (1 + \text{t} + \text{sum\_list}\ \text{ts}).
\end{equation*}

The cost for a function application consists of the cost of the function evaluation and the cost of evaluating the arguments.
We start with the cost for the function evaluation represented by evaluating the timing function.
By using the rule for primitive application of the evaluation semantic, we can conclude the following equation
\begin{equation*}
\text{vs, conv } \phi \hvdash \text{(pFun sum)\$\$ } [  \text{Const (N 1), } \mathcal{T}\ \text{fe} ] \hrightarrow \text{ N (1 + t)}.
\end{equation*}
The needed premises can be derived from \ref{cor:Tfeeval} and the axiom for primitive functions. From here, we only need to use the definition of the cost function to gain a statement about the cost term of the function.
\begin{equation}
\text{vs, conv } \phi \hvdash \text{cost fe} \hrightarrow \text{ N (1 + t)}.
\end{equation}
In the next step, we show that calling the cost function results in the same as evaluating its expression.
For this, we need the evaluation of the arguments $\texttt{es}$.
The evaluated result $\texttt{vs}$ is already given for the step-counting evaluation in \ref{cor:args}.
Here, we can use the lemma $\texttt{eval\_count\_eval}$ stating that the result is the same for the normal semantic.
\begin{equation*}
\forall \text{ i<length vs. } \rho \text{, } \phi \hvdash \text{es ! i } \hrightarrow \text{ vs ! i}
\end{equation*}
From this, we use the lemma eval\_no\_time\_trans with \ref{cor:notime} and can conclude that the result stays the same when the function definitions are converted.
\begin{equation} \label{eq:func_args}
\forall \text{ i<length vs. } \rho \text{, conv}\ \phi \hvdash \text{es ! i } \hrightarrow \text{ vs ! i}
\end{equation}

From \ref{cor:phi} and \ref{cor:notime}, we can conclude that f is an identifier for timing functions. Therefore
\begin{equation*}
  \text{conv } \phi \text{ f = Some fe}
\end{equation*}
holds.
Using this with \ref{cor:len} and the equation \ref{eq:func_args}, we can use the rule for function application to gain the wanted intermediate step
\begin{equation} \label{eq:func_id}
  \rho \text{, conv }\phi \hvdash \text{ cFun fn\$ es } \hrightarrow \text{ N (1 + t).}
\end{equation}

The next step involves combining \ref{eq:func_id} into one list with the cost evaluation of the arguments.
From \ref{cor:args} with \ref{cor:len} and \ref{cor:notime} the mapping of the arguments to their cost can be deducted.
\begin{equation*}
  \forall \text{i<length (map } \mathcal{T}\text{ es). }\rho\text{, conv }\phi \hvdash\text{ (map }\mathcal{T}\text{ es) ! i }\hrightarrow \text{ (map N ts) ! i}
\end{equation*}
Combined with equation \ref{eq:func_id}, we get the full list of arguments for the final sum function.
\begin{equation*}
  \begin{aligned}
  &\forall \text{i<length (cFun fn\$ es \# map } \mathcal{T}\text{ es). } \\
  &\rho\text{, conv }\phi \hvdash\text{ (cFun fn\$ es \# map }\mathcal{T}\text{ es) ! i }\hrightarrow \text{ ((N (1 + t)) \# (map N ts)) ! i}
  \end{aligned}
\end{equation*}
With this, \ref{cor:len} and the axiom for sum, we can now use the rule for primitive function application and show the wanted result.
\begin{equation*}
  \rho \text{, conv } \phi \hvdash \mathcal{T}\text{ (f \$ es) }\hrightarrow\text{ N (1 + t + sum\_list ts)}
\end{equation*}

This completes the proof for the function application case.
