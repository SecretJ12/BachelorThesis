

\section{Commands}\label{chapter:commands}

The implementation provides three commands.
The main one is $\texttt{define\_time\_fun}$.
It accepts the name of a function as argument, converts it and registers its running time function.
Additionally it tries to proof the termination automatically.
Details on this proof can be found in section \ref{chapter:termination}.

Although the termination proof should work automatically for most functions there are still edge cases, where it takes up too much time or even fails.
Therefore the command $\texttt{define\_time\_function}$ exists.
Similar to the command $\texttt{function}$ it only registeres the running time function.
Afterwards the user needs to prove termination manually using the command $\texttt{termination}$.

In order to convert mutual recursive functions, the name of all the related functions need to be provided.
This cannot be used to convert multiple not related functions at once.
Sometimes we also want to specify the used function equations explicitely.
This can be helpful for some cases as explained in \ref{chapter:nonconstant_zeros}.
To add the equations one needs to use the keyword $\texttt{equations}$ followed by the wanted equations.
The full command schema can be found in listing \ref{lst:cmds}.

The translation schema converts functions marked as zero differently, see section \ref{chapter:impl_schema}.
With the command $\texttt{define\_time\_0}$ more functions can me marked as zero.
This should be used with care and only for functions with a constant running time.
Section \ref{chapter:nonconstant_zeros} includes a discussion on that.
In listing \ref{lst:primitive} all default zero functions can be found.

\begin{lstlisting}[float,label=lst:cmds,caption=Schema of implemented command]
  define_time_fun {NameOfFunction}+ [equations {thm}+]
  define_time_function {NameOfFunction}+ [equations {thm}+]
  define_time_0 {NameOfFunction}
\end{lstlisting}

\begin{lstlisting}[float,label=lst:primitive,caption=Zero functions by default,mathescape=true]
  +, -, *, /, div, <, $\le$, Not, $\land$, $\lor$, =, Num.numeral_class.numeral
\end{lstlisting}

By default timing functions will be registered with the prefix ``$\texttt{T\_}$''.
To change this behaviour the configuration variable $\texttt{time\_prefix}$ can be adjusted.
Most of the time this should be not be changed in order avoid confusion and incompatibilities.

The converter will try to register the function without sequential mode first, which is the default of the $\texttt{function}$ command.
This behaviour is chosen as it does not change the given equations.
Therefore it avoids differences between the original and the timing function,
which would create problems for the automatic termination proof in some cases.
This will fail in some cases as for incomplete matching.
Then the converter falls back to sequential mode and prints out a warning.
