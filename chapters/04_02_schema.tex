
\section{Translation Schema} \label{chapter:impl_schema}

The translation schema of the converter is based on the schemas presented for first-order functions in \autoref{chapter:first_order}.
We use the schema by Nipkow \parencite{fds} for first-order functions as he provides a translation for all the needed cases.
All cases defined by Sands also equal this schema \parencite{sands} as they are equal to Nipkow's translation.
Therefore, the proof described in \autoref{chapter:formal} holds for first-order function translated by this schema.
In order to support higher-order functions, we extend the schema as described in \autoref{chapter:higher_order}.
Every argument that is a function will be replaced by a pair of the function and its timing function.
We disallow datatypes containing functions as parameters.
Refer to \autoref{chapter:restrictions} for more details.

Just like Nipkow and Sands, we treat some functions differently.
Here, those functions are called zero functions.
They include constructors and some basic mathematical operations and comparisons.
The user can mark additional function as zero function.
\autoref{chapter:commands} contains the corresponding command and a list of the functions marked as zero by default.
Only functions taking a constant amount of time should be marked as zero functions.
The user is in charge to mark only the correct functions.
\autoref{chapter:nonconstant_zeros} contains a discussion about this.

% C C C C C C C C C C C C C C C C C C C C C C C
We first define the function $\mathcal{C}$ transforming function definitions.
The converter differs between recursive and non-recursive functions.
Recursive functions will be translated with a leading $1+$ for the cost of the function call, while this is left out at non-recursive functions.
This is justified as the function call only represents a constant at non-recursive functions.
Therefore, the asymptotic running time class does not change.
It uses the function $\mathcal{T}$ to convert its expression into the timing version.
\autoref{fig:schema_C} defines the schema for $\mathcal{C}$.
\begin{figure}
  \begin{align*}
    \mathcal{C}\llbracket f\ a_1\ \dots\ a_n &= e\rrbracket = (T\_f\ a_{1}\ \dots\ a_{n} = \mathcal{T}\llbracket e\rrbracket) &&\text{| non-recursive}\\
    \mathcal{C}\llbracket f\ a_1\ \dots\ a_n &= e\rrbracket = (T\_f\ a_{1}\ \dots\ a_{n} = 1 + \mathcal{T}\llbracket e\rrbracket) &&\text{| recursive}
  \end{align*}
  \caption{Conversion of function definitions}
  \label{fig:schema_C}
\end{figure}

% T T T T T T T T T T T T T T T T T T T T T T
$\mathcal{T}$ is the main conversion function, defined in \autoref{fig:schema_T}.
It converts expressions just as defined by Nipkow.
The only exceptions are expressions that need to be evaluated normally.
For first-order functions, those expressions could be passed unchanged.
We use the function $\mathcal{N}$ to achieve this for higher-order functions.
This happens in the cases of $\texttt{if-else}$, $\texttt{case}$ and $\texttt{let}$.
Some optimizations are made to keep the translated functions simple in the same cases.
The $\texttt{if-else}$ and the $\texttt{case}$ expression are left out if all of their cases are empty.
Only the condition respectively the evaluated expression will be converted.
In the translation of the let expression, cases exist where the assigned variable is no longer used in the translated body.
In this case, we reduce the translation to this body.

\begin{figure}
\begin{align*}
  &\mathcal{T}\llbracket c\rrbracket &&= 0\\
  &\mathcal{T}\llbracket f\ a_{1}\ \dots\ a_{n}\rrbracket &&= \mathcal{F}\llbracket f\ a_{1}\ \dots\ a_{n}\rrbracket + \mathcal{T}\llbracket a_{1}\rrbracket + \dots + \mathcal{T}\llbracket a_{n}\rrbracket\\
  &\mathcal{T}\llbracket \text{if}\ c\ \text{then}\ et\ \text{else}\ ef\rrbracket &&= \mathcal{T}\llbracket c\rrbracket + (\text{if}\ \mathcal{N}\llbracket c\rrbracket\ \text{then}\ \mathcal{T}\llbracket et\rrbracket\ \text{else}\ \mathcal{T}\llbracket ef\rrbracket)\\
  &\mathcal{T}\llbracket \text{case}\ e\ \text{of}\ c_{1} \Rightarrow e_{1}\ |\ \dots\ |\ c_{n}\Rightarrow e_{n}\rrbracket &&= \mathcal{T}\llbracket e\rrbracket + \\
  & &&\ \ \ \ \ (\text{case}\ \mathcal{N}\llbracket e\rrbracket\ \text{of}\ c_{1}\Rightarrow\mathcal{T}\llbracket e_{1}\rrbracket\ |\ \dots\ |\ c_{n} \Rightarrow \mathcal{T}\llbracket e_{n}\rrbracket)\\
  &\mathcal{T}\llbracket \text{let}\ x = e_{1}\ \text{in}\ e_{2}\rrbracket &&= \mathcal{T}\llbracket e_{1}\rrbracket + (\text{let}\ x = \mathcal{N}\llbracket e_{1}\rrbracket\ \text{in}\ \mathcal{T}\llbracket e_{2}\rrbracket)
\end{align*}
  \caption{Main conversion schema for expressions}
  \label{fig:schema_T}
\end{figure}

% F F F F F F F F F F F F F F F F F F F F F F
The function $\mathcal{F}$ described in \autoref{fig:schema_F} handles the cost for function applications.
Applications to zero functions get translated to $0$ as their evaluation does not cost anything by definition.
Application to defined functions gets translated to the application of their timing function.
This timing function needs to be defined. Otherwise, the converter will throw an error.
All functions passed as arguments should also be translated to an application of their timing function.
As those arguments are now represented as a pair, we need to use the second element to receive the timing function.
In Sands' schema this step happens identically in the function described in \autoref{fig:higher_T}.
\begin{figure}
\begin{align*}
  \mathcal{F}\llbracket f\ a_{1}\ \dots\ a_{n}\rrbracket &= 0 &&\text{| Zero function}\\
  \mathcal{F}\llbracket f\ a_{1}\ \dots\ a_{n}\rrbracket &= (T\_f\ \mathcal{A}\llbracket a_{1}\rrbracket\ \dots\ \mathcal{A}\llbracket a_{n}\rrbracket) &&\text{| Defined function}\\
  \mathcal{F}\llbracket f\ a_{1}\ \dots\ a_{n}\rrbracket &= ((snd\ f)\ \mathcal{A}\llbracket a_{1}\rrbracket\ \dots\ \mathcal{A}\llbracket a_{n}\rrbracket) &&\text{| Passed function}
\end{align*}
\caption{Handling function applications}
\label{fig:schema_F}
\end{figure}

% A A A A A A A A A A A A A A A A A A A A A A
For first-order functions, the arguments for functions need to be passed unchanged.
We convert function arguments to pairs for higher-order functions.
Therefore, we must prepare the arguments before passing them.
Function $\mathcal{A}$ does this preparation.
All defined constants of non-primitive functions will be converted into the wanted pair.
For primitive functions, the second argument will be a lambda, taking the same number of arguments and returning $0$.
We do not need to change something for already passed functions, as they are pairs.
All the other expressions need to be evaluated as in the original function.
Again, we use the function $\mathcal{N}$ to achieve this behavior.
As we do not support curried functions, those expressions cannot evaluate to a function and, therefore, will not create problems.
The definition for $\mathcal{A}$ is described in \autoref{fig:schema_A}.
\begin{figure}
  \begin{align*}
    \mathcal{A}\llbracket f\rrbracket &= (f,\ T\_f) &&\text{| identifier of a defined non-primitive function}\\
    \mathcal{A}\llbracket f\rrbracket &= (f, (\lambda x\ \dots\ z.\ 0)) &&\text{| identifier of a defined primitive function}\\
    \mathcal{A}\llbracket f\rrbracket &= f &&\text{| passed function}\\
    \mathcal{A}\llbracket e\rrbracket &= \mathcal{N}\llbracket e\rrbracket &&\text{| other expressions}
  \end{align*}
  \caption{Preparing arguments for timing functions}
  \label{fig:schema_A}
\end{figure}

% N N N N N N N N N N N N N N N N N N N N N N
Lastly, we define the mentioned function $\mathcal{N}$.
This function replaces all occurrences of passed function $f$ by $\texttt{(fst f)}$.
As a result, we get an expression evaluating to the normal result as in the original function.
The definition can be found in \autoref{fig:schema_N}.
\begin{figure}
\begin{align*}
  \mathcal{N}\llbracket e_{1}\ e_{2}\rrbracket &= \mathcal{N}\llbracket e_{1}\rrbracket\ \mathcal{N}\llbracket e_{2}\rrbracket &&\text{| function application}\\
  \mathcal{N}\llbracket f\rrbracket &= (\texttt{fst}\ f) &&\text{| passed function}\\
  \mathcal{N}\llbracket c\rrbracket &= c &&\text{| constants / variables}
\end{align*}
\caption{Convert equation for normal evaluation}
\label{fig:schema_N}
\end{figure}

The functions $\mathcal{A}$ and $\mathcal{N}$ improve Sands' function $\mathcal{V}$ described in \autoref{fig:higher_V}.
Sands replaces all primitive and non-primitive function identifer by the described pair except for direct applications.
With this conversion, he first registers a helper function that evaluates as the original function but accepts the pairs instead of functions as arguments.
Through our function $\mathcal{N}$, we do not pass pairs to non-timing functions but the correct function argument.
We use $\mathcal{A}$ to pass functions as pairs only for timing functions.
We, therefore, have the advantage of not having to register an additional helper function.

As an example, we look at the default map function defined in Listing 2.3. At first, we see that the function map is recursive.
Therefore, the calling cost is 1.
The first case is then straightforward as it only contains the constant of the empty list.
In the second case, we must use all our defined functions to deal with the passed function f.
The application onto the Cons function is free because Cons is a constructor.
The translation is as follows.
\begin{lstlisting}[mathescape=true,language=translation]
  $\mathcal{C}\llbracket$map f [] = []$\rrbracket$
  $\equiv$ T_map f [] = 1 + $\mathcal{T}\llbracket$[]$\rrbracket$
  $\equiv$ T_map f [] = 1

  $\mathcal{C}\llbracket$map f (x#xs) = Cons (f x) (map f xs)$\rrbracket$
  $\equiv$ T_map f (x#xs) = 1 + $\mathcal{T}\llbracket$Cons (f x) (map f xs)$\rrbracket$
  $\equiv$ T_map f (x#xs) = 1 + $\mathcal{F}\llbracket$Cons (f x) (map f xs)$\rrbracket$ + $\mathcal{T}\llbracket$f x$\rrbracket$ + $\mathcal{T}\llbracket$map f xs$\rrbracket$
  $\equiv$ T_map f (x#xs) = 1 + $\mathcal{T}\llbracket$f x$\rrbracket$ + $\mathcal{T}\llbracket$map f xs$\rrbracket$
  $\equiv$ T_map f (x#xs) = 1 + $\mathcal{F}\llbracket$f x$\rrbracket$ + $\mathcal{T}\llbracket$x$\rrbracket$ + $\mathcal{F}\llbracket$map f xs$\rrbracket$ + $\mathcal{T}\llbracket$f$\rrbracket$ + $\mathcal{T}\llbracket$xs$\rrbracket$
  $\equiv$ T_map f (x#xs) = 1 + $\mathcal{F}\llbracket$f x$\rrbracket$ + $\mathcal{F}\llbracket$map f xs$\rrbracket$
  $\equiv$ T_map f (x#xs) = 1 + ((snd f) $\mathcal{A}\llbracket$x$\rrbracket$) + T_map $\mathcal{A}\llbracket$f$\rrbracket$ $\mathcal{A}\llbracket$xs$\rrbracket$
  $\equiv$ T_map f (x#xs) = 1 + ((snd f) $\mathcal{N}\llbracket$x$\rrbracket$) + T_map $\mathcal{N}\llbracket$f$\rrbracket$ $\mathcal{N}\llbracket$xs$\rrbracket$
  $\equiv$ T_map f (x#xs) = 1 + ((snd f) x) + T_map f xs
\end{lstlisting}

The result of this translation is similar to the translation Nipkow gives but with the passed function part of the pair.
The type of the timing function is
\begin{lstlisting}[language=translation,mathescape=true,keepspaces]
  ('a => 'b) $\times$ ('a => nat) => 'a list => nat.
\end{lstlisting}

In the next example, we translate the function $\texttt{Suc\_all}$ to see how a function is passed to a timing function.
The function is defined in \autoref{lst:schema_Suc_all}.

\begin{lstlisting}[language=isabelle,caption=Example function increasing every element in list of natural numbers,label=lst:schema_Suc_all]
fun Suc_all :: nat list => nat list where
  Suc_all xs = map Suc xs
\end{lstlisting}

$\texttt{Suc\_all}$ is a non-recursive function, meaning the calling cost of it is free.
The function $\texttt{T\_map}$ is already defined
We only need to prepare the function $\texttt{Suc}$ for passing.
$\texttt{Suc}$ is a constructor and, therefore, a zero function.
As a result, we represent the timing function of it by a lambda returning $0$.
The translation takes place in the following steps.
\begin{lstlisting}[mathescape=true,language=translation]
  $\mathcal{C}\llbracket$Suc_all xs = map Suc xs$\rrbracket$
  $\equiv$ T_Suc_all xs = $\mathcal{T}\llbracket$map Suc xs$\rrbracket$
  $\equiv$ T_Suc_all xs = $\mathcal{F}\llbracket$map Suc xs$\rrbracket$ + $\mathcal{T}\llbracket$Suc$\rrbracket$ + $\mathcal{T}\llbracket$xs$\rrbracket$
  $\equiv$ T_Suc_all xs = T_map $\mathcal{A}\llbracket$Suc$\rrbracket$ $\mathcal{A}\llbracket$xs$\rrbracket$
  $\equiv$ T_Suc_all xs = T_map (Suc, ($\lambda$x. 0)) $\mathcal{N}\llbracket$xs$\rrbracket$
  $\equiv$ T_Suc_all xs = T_map (Suc, ($\lambda$x. 0)) xs
\end{lstlisting}
