
\section{Restrictions} \label{chapter:restrictions}

Converting functions to their timing function starts simple if it is restricted to simple functions.
However, as always, supporting more and more functions makes the converter's code bigger and more challenging to read.
In order to keep the code maintainable, some restrictions were set.
The following section is going to explain and justify the current restrictions.

\subsection{Functions in datatypes}
As described earlier, there is a translation for functions that are passed as arguments.
Extending this for functions contained in pairs is straightforward, as the datatypes inside of pairs are not fixed.
The converter supports this automatically.
For arbitrary datatypes, this no longer holds.
For example, imagine a datatype with a constructor taking a function.
We cannot change the argument to a pair of functions as the datatype already fixes it.
Therefore, creating a new datatype taking the mentioned type would be needed.
As creating new datatypes is not in the wanted scope of this converter, it is not supported.

\subsection{Operations on datatypes} \label{chapter:nonconstant_zeros}
Most basic operations, such as the equals operator ``='', are marked as zero functions.
We can easily argue that a comparison can be made in constant time for a simple datatype such as nat.
This no longer holds for slightly more complicated types as lists.
The exact time for comparing two lists would at least be linear through the length of the list.
The command would need to register a timing version for every datatype the operator could be used with.
In order to support all datatypes, this had to happen in the $\texttt{datatype}$ command itself.
Therefore, it is the user's responsibility to only use those zero operators in timing functions if the constant time can be justified.
This would be the case for the equality operator if one of the sides is a constant as the empty list.

Also, the $\texttt{length}$ function for lists is part of this problem.
The function is an abbreviation of the size operator, which is automatically created for each datatype.
Converting it directly will create problems.
Instead, we can use size equations created for the list datatype.
Specifying them in the command will give us the desired result.
\autoref{lst:restr_length} shows the full command.
\begin{lstlisting}[language=isabelle,float,label=lst:restr_length,caption=Converting the length function correctly]
  define_time_fun length equations list.size(3) list.size(4)
\end{lstlisting}

\subsection{Partial application}
As described in \autoref{chapter:rel_curried}, Sands also proposes a translation schema for curried functions.
This schema cannot be used here as Isabelle uses a strict type system.
Trying to define Sands' $\texttt{app'}$ function as defined in \autoref{lst:curried_app'} will fail in Isabelle.
This is the case because, inside the then branch, the outcome of the func part is returned, while the else branch returns a triple.
However, the function $\texttt{capp}$ defined in \autoref{lst:curried_capp} can be registered.
However, as the arity counter is not coupled to the timing function itself, the timing function always needs to be of the form $\texttt{'a $\Rightarrow$ nat}$.
This type is not a wanted behavior, as we also want to evaluate the cost of a function with more than one argument left.

To overcome this issue, we would need to couple the counter to the number of arguments more closely.
As this would involve a more complex datatype, this is outside the scope of the converter.
