
\section{Restrictions} \label{chapter:restrictions}
The converter has some known restrictions.

\subsection{Functions in datatypes}
As described earlier there is a translations for functions being passed as arguments.
Extending this for functions contained in pairs is straight forward, as function type inside a pair can be changed to the needed pair.
The converter supports this automatically.
For arbitrary datatypes this no longer holds.
For example imagine a datatype with a constructor taking a function.
We cannot change the argument to a pair of functions as it is already fixed by the datatype.
Therefore it would be needed create a new datatype taking the mentioned type.
This is no longer in the wanted scope of this command.

\subsection{Operations on datatypes} \label{chapter:nonconstant_zeros}
Most basic operations as the equals operator ``='' are marked as zero function.
For simple datatype as nat we can easily argue that a comparsion can be made in constant time.
This no longer holds for slightly more complicated types as lists.
The exact time for a list would be linear times the time for the type of the contained items in the list.
The command would need to register a timing version for every term the operator could be used with.
Considering fully specified types this would be quite complex, without specified types it's not possible.
Therefore it is the users responsibility to only use those zero operators in timing functions if the constant time can be justified.
For lists this could be the case if one of the comparison sides is a constant as the empty list.

\subsection{Partial application}

Version proposed by Sands does not work in Isabelle \#TODO
