
\section{Examples}
This chapter presents the translation of some interesting functions.
Every listing first contains the original function, followed by the conversion command and the resulting timing function.
The function $\texttt{dummy}$ represents an arbitrary function.
We use it in some examples to show more interesting edge cases.

The first example shows that a $\texttt{if-else}$ expression is dropped if both cases are free of cost.
Only the converted condition is left.
\begin{lstlisting}[language=isabelle,mathescape=true]
fun bool_to_nat :: bool => nat where
  bool_to_nat b = (if dummy b then 1 else 0)
define_time_fun bool_to_nat

fun T_bool_to_nat :: bool => nat where
  T_bool_to_nat b = T_dummy b
\end{lstlisting}

The same happens if we define the same function with a case construct.
\begin{lstlisting}[language=isabelle,mathescape=true]
fun bool_to_nat_case :: bool => nat where
  bool_to_nat_case b = (case dummy b of True => 1 | False => 0)
define_time_fun bool_to_nat_case

fun T_bool_to_nat_case :: bool => nat where
    T_bool_to_nat_case b = T_dummy b
\end{lstlisting}

A similar optimization happens for let expressions.
In the following example, the converted function no longer uses the assigned variable of the inner $\texttt{let}$ expression.
Therefore, we reduce the expression to the body.
The outer $\texttt{let}$ is converted as expected.
\begin{lstlisting}[language=isabelle,mathescape=true]
fun let_add :: int => int => int where
  let_add x y = (let b = dummy y; a = dummy x in a + dummy b)
define_time_fun let_add

fun T_let_add :: int => int => nat where
  T_let_add x y = T_dummy y + (let b = dummy y in T_dummy x + T_dummy b)
\end{lstlisting}

If we want to convert a mutually recursive function, we need to provide the names of all related functions as specified in \autoref{chapter:commands}.
All functions will be translated as recursive.
\begin{lstlisting}[language=isabelle,mathescape=true]
fun even :: nat => bool
  and odd :: nat => bool where
  even 0 = True
| odd 0 = False
| even (Suc n) = odd n
| odd (Suc n) = even n
define_time_fun even odd

fun T_even_T_odd :: nat => nat where
  T_even 0 = 1
| T_odd 0 = 1
| T_even (Suc n) = 1 + T_odd n
| T_odd (Suc n) = 1 + T_even n
\end{lstlisting}

The next example shows how powerful the termination proof is.
First, we define the function calculating the collatz number.
Sadly, we cannot prove the termination of this function, but we can assume it using sorry.
Based on this assumption, the converter automatically proves the timing function's termination.
Therefore, the command uses the proof over dom presented in \autoref{chapter:termination}.
We receive a warning in the output.
\begin{lstlisting}[language=isabelle,mathescape=true,keepspaces]
function collatz :: nat =>nat where
  collatz n = (if n $\le$ 1 then 0 else
               if even n then 1 + collatz (n div 2)
               else 1 + collatz (3 * n + 1))
  by auto
termination sorry
define_time_0 "(dvd)"
define_time_fun collatz

fun T_collatz :: nat =>nat where
T_collatz n = 1 + (if n $\le$ 1 then 0 else
                   if even n then T_collatz (n div 2)
                   else T_collatz (3 * n + 1))
\end{lstlisting}

We now demonstrate how to use the $\texttt{define\_time\_function}$ command.
After applying it, we can prove termination using the $\texttt{termination}$ command.

\begin{lstlisting}[language=isabelle,mathescape=true]
function sum :: nat => nat => nat where
  sum i j = (if j $\le$ i then 0 else i + sum (Suc i) j)
  by pat_completeness auto
termination
  by (relation measure ($\lambda$(i,j). j - i)) auto
define_time_function sum
termination
  by (relation measure ($\lambda$(i,j). j - i)) auto

fun T_sum :: nat => nat => nat where
    T_sum i j = 1 + (if j $\le$ i then 0 else T_sum (Suc i) j)
\end{lstlisting}
