% !TeX root = ../main.tex
% Add the above to each chapter to make compiling the PDF easier in some editors.

\chapter{Implementation}\label{chapter:impl}

The main work of this thesis was the implementation of an automatic converter of functions to their running time function for Isabelle.
In order to keep the converter readable, it is restricted to non-curried functions.
The implementation works for the newest developer version of Isabelle and will be part of the Isabelle distribution.
The source code and more examples is in the associated GitHub repository \cite{repo}.

This chapter explains the usage and the behavior of the automatic converter.
\autoref{chapter:commands} shows the form of the added commands.
In the next \autoref{chapter:impl_schema}, we look at the conversion schema used by the converter.
In order to provide a pleasing user experience, an automatic termination proof exists for the running time function.
The details can be found in \autoref{chapter:termination}.
Defining running time functions can get complicated, so some restrictions exist to keep the implementation readable.
\autoref{chapter:restrictions} describes those restrictions.



\section{Commands}\label{chapter:commands}

The implementation provides three commands.
The main one is $\texttt{define\_time\_fun}$.
It accepts the name of a function as argument, converts it and registers its running time function.
Additionally it tries to proof the termination automatically.
Details on this proof can be found in \autoref{chapter:termination}.

Although the termination proof should work automatically for most functions there are still edge cases, where it takes up too much time or even fails.
Therefore the command $\texttt{define\_time\_function}$ exists.
Similar to the command $\texttt{function}$ it only registeres the running time function.
Afterwards the user needs to prove termination manually using the command $\texttt{termination}$.

In order to convert mutual recursive functions, the name of all the related functions need to be provided.
This cannot be used to convert multiple not related functions at once.
Sometimes we also want to specify the used function equations explicitely.
This can be helpful for some cases as explained in \autoref{chapter:nonconstant_zeros}.
To add the equations one needs to use the keyword $\texttt{equations}$ followed by the wanted equations.
The full command schema can be found in \autoref{lst:cmds}.

The translation schema converts functions marked as zero differently, see \autoref{chapter:impl_schema}.
With the command $\texttt{define\_time\_0}$ more functions can me marked as zero.
This should be used with care and only for functions with a constant running time.
\autoref{chapter:nonconstant_zeros} includes a discussion on that.
In \autoref{lst:primitive} all default zero functions can be found.

\begin{lstlisting}[float,label=lst:cmds,caption=Schema of implemented command]
  define_time_fun {NameOfFunction}+ [equations {thm}+]
  define_time_function {NameOfFunction}+ [equations {thm}+]
  define_time_0 {NameOfFunction}
\end{lstlisting}

\begin{lstlisting}[float,label=lst:primitive,caption=Zero functions by default,mathescape=true]
  +, -, *, /, div, <, $\le$, Not, $\land$, $\lor$, =, Num.numeral_class.numeral
\end{lstlisting}

By default timing functions will be registered with the prefix ``$\texttt{T\_}$''.
To change this behaviour the configuration variable $\texttt{time\_prefix}$ can be adjusted.
Most of the time this should be not be changed in order avoid confusion and incompatibilities.

The converter will try to register the function without sequential mode first, which is the default of the $\texttt{function}$ command.
This behaviour is chosen as it does not change the given equations.
Therefore it avoids differences between the original and the timing function,
which would create problems for the automatic termination proof in some cases.
This will fail in some cases as for incomplete matching.
Then the converter falls back to sequential mode and prints out a warning.


\section{Translation Schema} \label{chapter:impl_schema}

The translation schema of the converter is based on the schemas presented for first-order functions in \autoref{chapter:first_order}.
We use the schema by Nipkow \parencite{fds} for first-order functions as he provides a translation for all the needed cases.
All cases defined by Sands also equal this schema \parencite{sands} as they are equal to Nipkow's translation.
Therefore, the proof described in \autoref{chapter:formal} holds for this restricted part of the schema.
In order to support higher-order functions, we extend the schema as described in \autoref{chapter:higher_order}.
Every argument being a function will be replaced by a pair of the function and its timing function.

Just like Nipkow and Sands, we treat some functions differently.
Here, those functions are called zero functions.
They include constructors and some basic mathematical operations and comparisons.
Additionally, the user can mark any function as zero function.
\autoref{chapter:commands} contains the command for this and all the functions marked as zero by default.
Only functions taking a constant amount of time should be marked as zero functions.
The user is obliged to mark only the correct functions.
\autoref{chapter:nonconstant_zeros} contains a discussion about this.

% C C C C C C C C C C C C C C C C C C C C C C C
We first define the function $\mathcal{C}$ transforming function definitions.
The converter differs between recursive and non-recursive functions.
Recursive functions will be translated with a leading $1+$, while this is left out at non-recursive functions.
This is justified as the function call only represents a constant at non-recursive functions.
Therefore, the asymptotic running time class does not change.
It uses the function $\mathcal{T}$ to convert its expression into the timing version.
\autoref{fig:schema_C} defines the schema for $\mathcal{C}$.
\begin{figure}
  \begin{align*}
    \mathcal{C}\llbracket f\ a_1\ \dots\ a_n &= e\rrbracket = (T\_f\ a_{1}\ \dots\ a_{n} = \mathcal{T}\llbracket e\rrbracket) &&\text{| non-recursive}\\
    \mathcal{C}\llbracket f\ a_1\ \dots\ a_n &= e\rrbracket = (T\_f\ a_{1}\ \dots\ a_{n} = 1 + \mathcal{T}\llbracket e\rrbracket) &&\text{| recursive}
  \end{align*}
  \caption{Conversion of function definitions}
  \label{fig:schema_C}
\end{figure}

% T T T T T T T T T T T T T T T T T T T T T T
$\mathcal{T}$ is the main conversion function, defined in \autoref{fig:schema_T}.
It converts expressions just as defined by Nipkow.
The only exceptions are expressions that need to be evaluated normally.
For first-order functions, those expressions could be passed unchanged.
We use the function $\mathcal{N}$ to achieve this for higher-order functions.
This happens in the cases of $\texttt{if-else}$, $\texttt{case}$ and $\texttt{let}$.
\begin{figure}
\begin{align*}
  &\mathcal{T}\llbracket c\rrbracket &&= 0\\
  &\mathcal{T}\llbracket f\ a_{1}\ \dots\ a_{n}\rrbracket &&= \mathcal{F}\llbracket T\_f\ a_{1}\ \dots\ a_{n}\rrbracket + \mathcal{T}\llbracket a_{1}\rrbracket + \dots + \mathcal{T}\llbracket a_{n}\rrbracket\\
  &\mathcal{T}\llbracket \text{if}\ c\ \text{then}\ et\ \text{else}\ ef\rrbracket &&= \mathcal{T}\llbracket c\rrbracket + (\text{if}\ \mathcal{N}\llbracket c\rrbracket\ \text{then}\ \mathcal{T}\llbracket et\rrbracket\ \text{else}\ \mathcal{T}\llbracket ef\rrbracket)\\
  &\mathcal{T}\llbracket \text{case}\ e\ \text{of}\ c_{1} \Rightarrow e_{1}\ |\ \dots\ |\ c_{n}\Rightarrow e_{n}\rrbracket &&= \mathcal{T}\llbracket e\rrbracket + \\
  & &&\ \ \ \ \ (\text{case}\ \mathcal{N}\llbracket e\rrbracket\ \text{of}\ c_{1}\Rightarrow\mathcal{T}\llbracket e_{1}\rrbracket\ |\ \dots\ |\ c_{n} \Rightarrow \mathcal{T}\llbracket e_{n}\rrbracket)\\
  &\mathcal{T}\llbracket \text{let}\ x = e_{1}\ \text{in}\ e_{2}\rrbracket &&= \mathcal{T}\llbracket e_{1}\rrbracket + (\text{let}\ x = \mathcal{N}\llbracket e_{1}\rrbracket\ \text{in}\ \mathcal{T}\llbracket e_{2}\rrbracket)
\end{align*}
  \caption{Main conversion schema for expressions}
  \label{fig:schema_T}
\end{figure}

% F F F F F F F F F F F F F F F F F F F F F F
The function $\mathcal{F}$ handles the cost for function applications.
Applications to zero functions get translated to $0$ as their evaluation does not cost anything by definition.
Application to defined functions gets translated to the application of their timing function.
This timing function needs to be defined. Otherwise, the converter will throw an error.
All functions passed as arguments should also be translated to an application of their timing function.
As those arguments are now represented as a pair, we need to use the second element to receive the timing function.
The schema is defined in \autoref{fig:schema_F}.
\begin{figure}
\begin{align*}
  \mathcal{F}\llbracket f\ a_{1}\ \dots\ a_{n}\rrbracket &= 0 &&\text{| Zero function}\\
  \mathcal{F}\llbracket f\ a_{1}\ \dots\ a_{n}\rrbracket &= (T\_f\ \mathcal{A}\llbracket a_{1}\rrbracket\ \dots\ \mathcal{A}\llbracket a_{n}\rrbracket) &&\text{| Defined function}\\
  \mathcal{F}\llbracket f\ a_{1}\ \dots\ a_{n}\rrbracket &= ((snd\ f)\ \mathcal{A}\llbracket a_{1}\rrbracket\ \dots\ \mathcal{A}\llbracket a_{n}\rrbracket) &&\text{| Passed function}
\end{align*}
\caption{Handling function applications}
\label{fig:schema_F}
\end{figure}

% A A A A A A A A A A A A A A A A A A A A A A
For first-order functions, the arguments for functions need to be passed unchanged.
As we get functions as pairs for higher-order functions, we must prepare the arguments before passing them.
Function $\mathcal{A}$. does this preparation.
All defined constants of non-primitive functions will be converted into the wanted pair.
For primitive functions, the second argument will be a lambda, taking the same number of arguments and returning $0$.
We do not need to change something for passed functions, as they are already pairs.
All the other expressions need to be evaluated as in the original function.
Again, we use the function $\mathcal{N}$ to achieve this behavior.
As we do not support curried functions, those expressions cannot evaluate to a function and, therefore, will not create problems.
The definition for $\mathcal{A}$ is described in \autoref{fig:schema_A}.
\begin{figure}
  \begin{align*}
    \mathcal{A}\llbracket f\rrbracket &= (f,\ T\_f) &&\text{| identifier of a defined non-primitive function}\\
    \mathcal{A}\llbracket f\rrbracket &= (f, (\lambda x\ \dots\ z.\ 0)) &&\text{| identifier of a defined primitive function}\\
    \mathcal{A}\llbracket f\rrbracket &= f &&\text{| passed function}\\
    \mathcal{A}\llbracket e\rrbracket &= \mathcal{N}\llbracket e\rrbracket &&\text{| other expressions}
  \end{align*}
  \caption{Preparing arguments for timing functions}
  \label{fig:schema_A}
\end{figure}

% N N N N N N N N N N N N N N N N N N N N N N
Lastly, we define the mentioned function $\mathcal{N}$.
This function replaces all occurrences of passed function $f$ by $\texttt{(fst f)}$.
As a result, we get an expression evaluating to the normal result as in the original function.
The definition can be found in \autoref{fig:schema_N}.
\begin{figure}
\begin{align*}
  \mathcal{N}\llbracket e_{1}\ e_{2}\rrbracket &= \mathcal{N}\llbracket e_{1}\rrbracket\ \mathcal{N}\llbracket e_{2}\rrbracket &&\text{| function application}\\
  \mathcal{N}\llbracket f\rrbracket &= (\texttt{fst}\ f) &&\text{| passed function}\\
  \mathcal{N}\llbracket c\rrbracket &= c &&\text{| constants / variables}
\end{align*}
\caption{Convert equation for normal evaluation}
\label{fig:schema_N}
\end{figure}

The functions $\mathcal{A}$ and $\mathcal{N}$ are similar to Sands' function $\mathcal{V}$ described in \autoref{fig:higher_V}.
Sands replaces all primitive and non-primitive function identifer by the described pair.
Our functions only do this in the needed places.
Additionally, he uses the first argument of the pair for function applications similar to our function $\mathcal{F}$.

As an example, we look at the default map function defined in Listing 2.3. At first, we see that the function map is recursive.
Therefore, the calling cost is 1.
The first case is then straightforward as it only contains the constant of the empty list.
In the second case, we must use all our defined functions to deal with the passed function f.
The application onto the Cons function is free because Cons is a constructer.
The translation looks the following:
\begin{lstlisting}[mathescape=true,language=translation]
  $\mathcal{C}\llbracket$map f [] = []$\rrbracket$
  $\equiv$ T_map f [] = 1 + $\mathcal{T}\llbracket$[]$\rrbracket$
  $\equiv$ T_map f [] = 1

  $\mathcal{C}\llbracket$map f (x#xs) = f x # map f x$\rrbracket$
  $\equiv$ T_map f (x#xs) = 1 + $\mathcal{T}\llbracket$f x # map f x$\rrbracket$
  $\equiv$ T_map f (x#xs) = 1 + $\mathcal{F}\llbracket$Cons (f x) (map f x)$\rrbracket$ + $\mathcal{T}\llbracket$f x$\rrbracket$ + $\mathcal{T}\llbracket$map f xs$\rrbracket$
  $\equiv$ T_map f (x#xs) = 1 + $\mathcal{T}\llbracket$f x$\rrbracket$ + $\mathcal{T}\llbracket$map f xs$\rrbracket$
  $\equiv$ T_map f (x#xs) = 1 + $\mathcal{F}\llbracket$f x$\rrbracket$ + $\mathcal{T}\llbracket$x$\rrbracket$ + $\mathcal{F}\llbracket$map f xs$\rrbracket$ + $\mathcal{T}\llbracket$f$\rrbracket$ + $\mathcal{T}\llbracket$xs$\rrbracket$
  $\equiv$ T_map f (x#xs) = 1 + $\mathcal{F}\llbracket$f x$\rrbracket$ + $\mathcal{F}\llbracket$map f xs$\rrbracket$
  $\equiv$ T_map f (x#xs) = 1 + ((snd f) $\mathcal{A}\llbracket$x$\rrbracket$) + T_map $\mathcal{A}\llbracket$f$\rrbracket$ $\mathcal{A}\llbracket$xs$\rrbracket$
  $\equiv$ T_map f (x#xs) = 1 + ((snd f) $\mathcal{N}\llbracket$x$\rrbracket$) + T_map $\mathcal{N}\llbracket$f$\rrbracket$ $\mathcal{N}\llbracket$xs$\rrbracket$
  $\equiv$ T_map f (x#xs) = 1 + ((snd f) x) + T_map f xs
\end{lstlisting}

The result of this translation is similar to the translation Nipkow gives but with the passed function part of the pair.
In the next example, we translate the function $\texttt{Suc\_all}$ to see how a function is passed to a timing function.
The function is defined in \autoref{lst:schema_Suc_all}.

\begin{lstlisting}[language=isabelle,caption=Example function increasing every element in list of natural numbers,label=lst:schema_Suc_all]
fun Suc_all :: nat list => nat list where
  Suc_all xs = map Suc xs
\end{lstlisting}

$\texttt{Suc\_all}$ is a non-recursive function, meaning the calling cost of it is free.
The function $\texttt{T\_map}$ is already defined
We only need to prepare the function $\texttt{Suc}$ for passing.
$\texttt{Suc}$ is a constructor and, therefore, a zero function.
As a result, we represent the timing function of it by a lambda returning $0$.
The translation has the following steps:
\begin{lstlisting}[mathescape=true,language=translation]
  $\mathcal{C}\llbracket$Suc_all xs = map Suc xs$\rrbracket$
  $\equiv$ T_Suc_all xs = $\mathcal{T}\llbracket$map Suc xs$\rrbracket$
  $\equiv$ T_Suc_all xs = $\mathcal{F}\llbracket$map Suc xs$\rrbracket$ + $\mathcal{T}\llbracket$Suc$\rrbracket$ + $\mathcal{T}\llbracket$xs$\rrbracket$
  $\equiv$ T_Suc_all xs = T_map $\mathcal{A}\llbracket$Suc$\rrbracket$ $\mathcal{A}\llbracket$xs$\rrbracket$
  $\equiv$ T_Suc_all xs = T_map (Suc, ($\lambda$x. 0)) $\mathcal{N}\llbracket$xs$\rrbracket$
  $\equiv$ T_Suc_all xs = T_map (Suc, ($\lambda$x. 0)) xs
\end{lstlisting}



\section{Termination proof} \label{chapter:termination}

The command $\texttt{define\_time\_fun}$ tries to automatically prove termination of the timing function.
Therefore, it uses two different tactics.
The first try equals the command $\texttt{fun}$ as the command name suggests.
Both use the tactic $\texttt{lexicographic\_order}$ in order to prove termination.
We now look at the following function $\texttt{sum}$, where this tactic fails.
\begin{lstlisting}[language=isabelle,mathescape=true]
  function sum :: "nat => nat => nat" where
    "sum i j = (if j $\le$ i then 0 else i + sum (Suc i) j)"
    by pat_completeness auto
  termination
    by (relation "measure ($\lambda$(i,j). j - i)") auto
\end{lstlisting}

Termination needs to be proved manually.
Therefore, the first tactic also fails for the timing function.
However, as we have already proved termination for this function, we can use it for the running time function.
The second strategy does this and tries to cover all functions.
In the first step, we register the timing function equivalent to using the $\texttt{function}$ command.
\begin{lstlisting}[language=isabelle,mathescape=true,caption=Function registration,label=lst:sum_reg]
  function (domintros) T_sum :: "nat => nat => nat" where
    "T_sum i j = 1 + (if j $\le$ i then 0 else T_sum (Suc i) j)"
    by pat_completeness auto
\end{lstlisting}
In Isabelle, every function needs to terminate.
Before this, the simp rules are not usable.
However, we receive another function called $T\_sum\_dom$.
It represents the domain of arguments in which $T\_sum$ terminates.
Therefore, it takes the arguments of $T\_sum$ as a tuple and yields $True$ if the function terminates for them and $False$ otherwise.
Based on this, the rules psimps are generated.
They state the following: Under the assumption of $T\_sum$ terminating, the corresponding simp rule holds.
The psimps rule for $T\_sum$ is given in equation \ref{eq:T_sum.psimps}.

\begin{equation}
  \begin{aligned}
  &\texttt{T\_sum\_dom (?i, ?j)}\\
  &\texttt{$\Longrightarrow$ T\_sum ?i ?j = 1 + (if ?j $\le$ ?i then 0 else T\_sum (Suc ?i) ?j)}
  \end{aligned}
  \label{eq:T_sum.psimps}
\end{equation}

From this equation, we can see how the termination proof works.
In order to obtain the simp rules, we need to show that $T\_sum\_dom$ holds for every argument.
Before we start with the proof, we need to look at another set of generated rules.
The $\texttt{domintros}$ rules state when a function call terminates.
Termination happens if all the recursive calls made also terminate.
Those rules are not generated by default due to performance reasons.
We needed to explicitly pass the option domintros to obtain them.
This already happened in the listing \ref{lst:sum_reg}.
For our sum function, the domintros rule has the following form shown in equation \ref{eq:domintros}.
From this, we can see that a function call with the variables $\texttt{i}$ and $\texttt{j}$ with $\texttt{j > i}$ terminates
if the function call with $\texttt{Suc i}$ and $\texttt{j}$ terminates.
\begin{equation}
  \texttt{($\lnot$ ?j $\le$ ?i $\Longrightarrow$ T\_sum\_dom (Suc ?i, ?j)) $\Longrightarrow$ T\_sum\_dom (?i, ?j)}
\label{eq:domintros}
\end{equation}

This gives us all the rules we need to prove our goal.
We start by setting up the goal of the form ``$\texttt{T\_f\_dom (a}_{1}\texttt{,}\dots\texttt{,a}_{n}\texttt{)}$''.
On this goal we perform an induction with the induction schema provided by the original function.
This is already the step where we use the termination proof of the original function,
as the induction schema is proved through the termination.
To argue about the next step, we need to look at the translation schema for our timing functions.
Taking the if-else construct as an example, the place where recursive function calls are made does not change.
All function calls inside the condition will still be executed without another precondition.
For the function calls inside the then and else branches, the preconditions stay the same, as the condition is evaluated as in the original function.
This justifies why the resulting cases stay close to the original function.
Taking the sum function as an example, the induction creates the goal
\begin{equation*}
  \texttt{$\bigwedge$i j. ($\lnot$ j $\le$ i $\Longrightarrow$ T\_sum\_dom (Suc i, j)) $\Longrightarrow$ T\_sum\_dom (i,j)}.
\end{equation*}
As expected, it is similar to the domintros rule shown in equation \ref{eq:domintros}.
With its help, we are also able to solve the goal.
In order to support as many cases as possible, we use metis as an advanced prover.
With the just-proven goal, the auto tactic can now prove termination.
The whole proof can be found in listing \ref{lst:proof_schema}.
\begin{lstlisting}[language=isabelle,mathescape=true,label=lst:proof_schema,caption=Proof schema over dom with help of original function]
  lemma T_sum_dom: "T_sum_dom (i,j)"
    apply (induction i j rule: sum.induct)
    apply (metis T_sum.domintros)
    done
  termination
    by (auto simp: T_sum_dom)
\end{lstlisting}

Internally, auto is used before metis, as it can do some more simplifications and, therefore, cover more edgecases.
For functions with multiple equations, the induction schema will create multiple goals.
The automation first tries to solve every goal by the corresponding domintros rule and falls back to all domintros rules in case of failure.
This behavior reduces the number of ``Unused theorems'' warnings.
The named lemma in listing \ref{lst:proof_schema} is just for demonstration purposes.
The converter works with only internally usable goals.


\section{Restrictions} \label{chapter:restrictions}
The converter has some known restrictions.

\subsection{Functions in datatypes}
As described earlier there is a translations for functions being passed as arguments.
Extending this for functions contained in pairs is straight forward, as function type inside a pair can be changed to the needed pair.
The converter supports this automatically.
For arbitrary datatypes this no longer holds.
For example imagine a datatype with a constructor taking a function.
We cannot change the argument to a pair of functions as it is already fixed by the datatype.
Therefore it would be needed create a new datatype taking the mentioned type.
This is no longer in the wanted scope of this command.

\subsection{Operations on datatypes} \label{chapter:nonconstant_zeros}
Most basic operations as the equals operator ``='' are marked as zero function.
For simple datatype as nat we can easily argue that a comparsion can be made in constant time.
This no longer holds for slightly more complicated types as lists.
The exact time for a list would be linear times the time for the type of the contained items in the list.
The command would need to register a timing version for every term the operator could be used with.
Considering fully specified types this would be quite complex, without specified types it's not possible.
Therefore it is the users responsibility to only use those zero operators in timing functions if the constant time can be justified.
For lists this could be the case if one of the comparison sides is a constant as the empty list.

\subsection{Partial application}

As described in chapter \ref{chapter:rel_curried} Sands also proposes a translation schema for curried functions.
This schema cannot be used as Isabelle uses a strict type system.
Trying to define the basic function $\texttt{app'}$ as defined in listing \ref{lst:curried_app'} in Isabelle will fail.
In the then branch the outcome of the func part is returned, while the else branch returns a triple.
The function $\texttt{capp}$ defined in listing \ref{lst:curried_capp} can be registered.
But as the arity counter is not coupled to the timing function itself, the timing function always needs to be of the form $\texttt{'a $\Rightarrow$ nat}$.
This is not a wanted behaviour, as we also want to evaluate the cost of a not fully evaluated function.

To overcome this issue we would need to couple the counter to the number of arguments more closely.
As this would involve a more complex datatype this is not in scope of the conversion command.


\section{Probably examples}
