% !TeX root = ../main.tex
% Add the above to each chapter to make compiling the PDF easier in some editors.

\chapter{Summary}\label{chapter:summary}

This work tries to automate the conversion of functions to their running time function in Isabelle.
Therefore we look at existing conversion schemas.
Naturally we start by restricting the function in the beginning and look at Non-Curried First Order Functions.
The conversion schema for this functions are quite common and similar in differnet papers.
For Higher Order Functions only Sands explains an extension for the existing schema.
He extends the given function to a tuple containing the function and the timing function.
In order to also deal with curried functions he extends this pair by a counter for the number of arguments left.
In the end we gain a schema able to deal with Higher Order Curried functions.
Sands provides prooves for all this schemas.
Here we only take a look at the correctness proof of the schema for First Order Functions.
The work provides a formalization of it in Isabelle and explains it.

In the main part an automatic converter for functions into their running time function in Isabelle is provided.
As the schema for Curried Function gets too complicated we restrict ourself to Non-Curried Higher Order Functions.
Additionally we don't allow functions to be passed in datatypes as conversion of it is no quite obvious and get laborious.
The used schema is based on the proposals by Sands \parencite{sands} and Nipkow \parencite{fds}.
As a result we gain a simple command able convert the restricted functions automatically.
It contains an automation for the termination proof based based on the $\texttt{lexicographic\_order}$ tactic first.
If this fails a more advanced schema is used to proof termination based on the termination of the original function.
This proof should covers most functions.
The command was added to the Isabelle source files and will be accessible in the next major verison.

The schema for curried function proposed by Sands cannot be used in Isabelle.
Therefore it is needed to think about an adaption of this schema to overcome the current limitation on non-curried functions.
Additionally there is a restrictions for functions as equality defined for and by every datatype.
Here we would need to provide a running time version specific for every datatype not just function.
As conversion is not hard once the wanted equations have been found, extending this is a rather technical task.
