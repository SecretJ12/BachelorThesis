% !TeX root = ../main.tex
% Add the above to each chapter to make compiling the PDF easier in some editors.

\chapter{Summary}\label{chapter:summary}
This work automates converting functions to their running time function in Isabelle.
Therefore, we looked at existing conversion schemas by Sands \parencite{sands} and Nipkow \parencite{fds}.
We presented a translation schema for non-curried first-order functions.
Afterward, we extended this schema for higher-order functions.
To deal with the functions as arguments, we extended them to pairs containing the function and the timing function.
To support curried functions, we extended this pair by a counter for the number of arguments left.
The next part looked at a correctness proof for the first-order functions schema.
Sands presented this proof for his schema, and we formalized it in Isabelle.

The main part presents the automatic converter for functions into their running time function in Isabelle.
We restricted the accepted functions to non-curried higher-order functions.
This keeps the code readable and maintainable.
The translation schema is based on the proposals by Sands \parencite{sands} and Nipkow \parencite{fds}.
We can use the command $\texttt{define\_time\_fun}$ to convert a function automatically.
It first tries to prove termination using the $\texttt{lexicographic\_order}$ tactic.
If this fails, we use a more advanced tactic that uses the termination proof of the original function.
This proof tactic covers most functions.
We can use the command $\texttt{define\_time\_function}$ to prove termination manually.

Although the converter works automatically, we need to use it with care.
Most basic mathematical operations and comparisons are marked as zero functions.
Those zero functions will assumed as constant and get translated to $0$.
The constant time may not hold in certain situations as the equality operator on lists.
We would need to extend the datatype command to create a timing function for every overload.
Also, curried functions are disallowed.
Sands proposed a schema to tackle these, but it does not work in Isabelle due to the strict type system.
We would need to consider adapting this schema to overcome this limitation.
Additionally, we restrict these functions' arguments.
Their type may not be datatypes that contain other functions, as conversion is not obvious and quickly becomes laborious.

The converter is part of the current Isabelle development version and will be part of the next major Isabelle version.
