% !TeX root = ../main.tex
% Add the above to each chapter to make compiling the PDF easier in some editors.

\chapter{Summary}\label{chapter:summary}

This work automates the conversion of functions to their running time function in Isabelle.
Therefore we looked at existing conversion schemas by Sands \parencite{sands} and Nipkow \parencite{fds}.
We presented a translation schema for non-curried first-order functions.
Afterwards we extended this schema for higher-order functions.
To deal with the functions as arguments we extended them to pairse containing the function and the timing function.
For the support of curried function we extended this pair by a counter for the number of arguments left.
The next part looked at a correctness proof for the first-order functions schema.
Sands presented this proof for his schema, and we provided a formalization of it in Isabelle.

The main part presents the automatic converter for functions into their running time function in Isabelle.
We restricted the accepted functions to non-curried higher-order functions.
This keeps the code readable and maintainable.
Additionally, we restrict these functions' arguments.
Their type may not be datatypes that contain other functions, as conversion is not obvious and quickly becomes laborious.
The used schema is based on the proposals by Sands \parencite{sands} and Nipkow \parencite{fds}.
We can now use the command $\texttt{define\_time\_fun}$ to convert a function automatically.
It first tries to proof termination using the $\texttt{lexicographic\_order}$ tactic first.
If this fails we use a more advance tactic that uses the termination proof of the original funcition.
This proof covers most functions.
We can use the command $\texttt{define\_time\_function}$ to prove termination manually.

Although the converter works automatically, we need to use it with care.
Most basic mathematical operations and comparisons are marked as zero functions.
Those zero functions will assumed as constant and get translated to $0$.
The constant time may not hold in certain situations as equality on lists.
We would need to extends the datatype command to create a timing function for every overload.
Also curried functions are not allowed.
Sands proposed a schema to tackle them, but it does not work in Isabelle due to the strict type styem.
To overcome this limmitation we would need to think about an adaption of this schema.

The converter was added to the Isabelle distribution and will be accessible in relase 2024.
